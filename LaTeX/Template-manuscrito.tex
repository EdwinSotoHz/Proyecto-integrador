\documentclass{IEEEcsmag}

\usepackage[colorlinks,urlcolor=blue,linkcolor=blue,citecolor=blue]{hyperref}
\expandafter\def\expandafter\UrlBreaks\expandafter{\UrlBreaks\do\/\do\*\do\-\do\~\do\'\do\"\do\-}
\usepackage{upmath,color}

\usepackage[spanish]{babel}
%\usepackage[latin1]{inputenc}
\usepackage[utf8]{inputenc}  

\jvol{1}
\jnum{1}
\paper{1}
\jmonth{Noviembre}
\jname{ITICs letters}
\jtitle{Proyectos Integradores}
\pubyear{2023}

\newtheorem{theorem}{Theorem}
\newtheorem{lemma}{Lemma}



\setcounter{secnumdepth}{0}

\begin{document}

\sptitle{Proyecto Integrador de Primer Semestre}

\title{Software de resolución de problemas de Ingeniería }

\author{Cuadros Romero Francisco Javier}
\affil{Instituto Tecnológico Superior del Occidente del Estado de Hidalgo, Mixquiahuala, Hgo., 42700, Mexico}

\author{Neri Pérez Giovany Humberto}
\affil{Instituto Tecnológico Superior del Occidente del Estado de Hidalgo, Mixquiahuala, Hgo., 42700, Mexico}

%\author{Third Author III}
%\affil{Institute, City, (State), Postal Code, Country}

\markboth{ITSOEH/ITICS/PROYECTO INTEGRADOR PRIMER SEMESTRE}{THEME/FEATURE/DEPARTMENT}

\begin{abstract}
Un resumen (abstract) es un párrafo único que resume los aspectos importantes del manuscrito. A menudo indica si el manuscrito es un informe de un trabajo nuevo, una revisión o una descripción general, o una combinación de ambos. No cite referencias en el resumen. Este tipo de documento debe incluir contenido propiedad de los autores; es decir, no debe contener contenido de otras fuentes, ademas la redacción debe  estar dirigida a un tipo de lector técnico general. Este archivo se encuentra disponible en \href{https://github.com/fcuadrosgithub/integrador-primero.git}{https://github.com/fcuadrosgithub/integrador-primero.git}.
\end{abstract}

\maketitle
\chapteri{L}a introducción debe proporcionar información general (incluidas referencias relevantes) y debe indicar el propósito del manuscrito. En esta sección describa de manera clara y precisa el objetivo del proyecto integrador, la metodología que piensa usar y los resultados obtenidos de manera muy general. Dentro de esta sección puede citar trabajos relevantes de otros si lo cree necesario.

Esta sección debe dar un panorama muy general al lector de cual es el problema a resolver, que metodología utilizó para dar solución al problema y cuales fueron los resultados obtenidos. 

La redacción del manuscrito debe ser en tercera persona y queda estrictamente prohibido el uso de palabras coloquiales o Español informal. En lugar de esto utilice un lenguaje formal que el mayor numero de personas pueda entender.

\section{COPYRIGHT Y ACCESO ABIERTO}

Una vez que los autores entreguen este documento para su evaluación también seden los derechos del contenido de este manuscrito a la carrera de Ingeniería en Tecnológicas de la Información y Comunicaciones (ITICs) del Instituto Tecnológico Superior del Occidente del Estado de Hidalgo (ITSOEH). Esto conlleva que la carrera puede usar el contenido de este articulo para efectos de difusión del quehacer de los estudiantes de la carrera o en cualquier otra actividad que la carrera considere pertinente. Cabe mencionar que en ningún momento el orden o los nombres de los autores sera modificado de ninguna manera y siempre se les dará el crédito correspondiente. 
\section{PROBLEMAS}
A continuación se describen los problemas que el equipo deberá resolver.
\begin{enumerate}
\item Dados 2 puntos $A \mbox{ y } B$ con coordenadas $x_{1}, y_{1}$ y $x_{2}, y_{2}$  respectivamente. Regresar la ecuación de la recta y el ángulo interno $\alpha$ que se forma entre el eje horizontal y la recta. 
%Por ejemplo con los puntos $A(2, 1)$ y $B(-3, 2)$ la ecuación debe ser $y = -\frac{1}{5}x + \frac{7}{5}$. 
\item Dada una ecuación cuadratica regresar los valores de las raíces en caso de que estén sobre el conjunto de los números reales, en caso contrario indicar que la solución esta en el conjunto de los números complejos. 
\item Dada una circunferencia con centro en el punto $C$ con coordenadas $(x_{1}, y_{1})$ y radio $r$, evaluar si un punto $T$ con coordenadas $(x_{2}, y_{2})$ esta dentro del area de la circunferencia.
\item Dado un numero decimal entero positivo o negativo regresar su equivalente en binario.
\item Dado un numero binario de $n$ bits regresar su equivalente en decimal.
\item Dada una tabla de verdad de $n$ bits generar la expresión booleana que genere de manera fidedigna las salidas de esta tabla.
\end{enumerate}

\section{Sección Problema 1}



\newpage
Contenido del primer problema...
\newpage


\section{Sección Problema 2}
Contenido del segundo problema...
\newpage
Contenido del segundo problema...
\newpage


\section{Sección Problema 3}
%Lizbeth

\begin{itemize}
    \item Descripción del problema:
\end{itemize}
\begin{enumerate}
El problema 3 busca evaluar si un punto $T$ con coordenadas $(x_{2}, y_{2})$ esta dentro del área de la circunferencia, sabiendo que el punto $C$ se encuentra en el centro de la misma. Lo cual solo el usuario ingresara las coordenadas $(x_{1}, y_{1})$ y $(x_{2}, y_{2})$  y el radio $r$ para realizar el proceso de la  ecuación que se ocupo, y presentar la respuesta.
\end{enumerate}


\begin{itemize}
    \item Definición de solución:
\end{itemize} 
La solución para evaluar si un punto $T$ con coordenadas $(x_{2}, y_{2})$ está dentro del área de una circunferencia con centro en el punto $C$ con coordenadas $(x_{1}, y_{1})$ y radio $r$ implica utilizar la fórmula de la distancia euclidiana, la distancia se calcula como la raíz cuadrada de la suma de los cuadrados de las diferencias entre las coordenadas x y las coordenadas y de ambos puntos:
distancia = 
\begin{equation}
   \sqrt{ (x_2 - x_1)^2 + (y_2 - y_1)^2 }     
\end{equation}



Al final solo comparamos la distancia calculada con el radio $r$ de la circunferencia:
\begin{enumerate}
\item Si la distancia es mayor que el radio $r$, el punto $T$ está fuera de la circunferencia.
\item Si la distancia es igual al radio $r$, el punto $T$ está en el borde de la circunferencia.
\item Si la distancia es menor que el radio $r$, el punto $T$ está dentro del área de la circunferencia.    
\end{enumerate}


\begin{itemize}
    \item Diseño de la solución:
\end{itemize}
\begin{enumerate} 
\item Solicitar al usuario que ingrese las coordenadas del punto $C$ y el punto $T$, es decir $(x_{1}, y_{1})$ y $(x_{2}, y_{2})$ y el radio $r$.
\item Calcular la distancia entre el centro y el punto.
\item Verificar si el punto está dentro del área de la circunferencia
\item Al final mostrar el resultado al usuario 
\end{enumerate}


\begin{itemize}
    \item Desarrollo de solución:
\end{itemize}
El algoritmo de solución del problema 3 comienza utilizando un objeto Scanner para leer las coordenadas del centro (punto $C$), el radio $r$ y las coordenadas del punto a verificar (punto $T$) desde la entrada estándar. Las coordenadas se ingresan en el formato $"x, y"$ separadas por una coma.
\begin{lstlisting}[style=javaStyle]
Scanner datC = new Scanner(System.in);
        System.out.print("""
                         Ingrese las coordenadas del punto C 
                         seperadas por una coma (x1, y1):
                         """);
    String[] puntoC = (datC.nextLine()).split(",");
        
    System.out.print("Ingrese el radio de la circunferencia: ");
    float r = datC.nextFloat();
    datC.nextLine();
    
    System.out.print("""
                         Ingrese las coordenadas del punto T
                         seperadas por una coma (x2, y2): 
                         """);
    String[] puntoT = (datC.nextLine()).split(",");
    datC.close();
\end{lstlisting}
Posteriormente se convierten los valores de los puntos $(x_{1}, y_{1})$ y $(x_{2}, y_{2})$. en valores enteros para ser utilizados  en la ecuación de la distancia:

\begin{lstlisting}[style=javaStyle]
    int x1 = Integer.parseInt(puntoC[0].trim());
    int y1 = Integer.parseInt(puntoC[1].trim());
        
    int x2 = Integer.parseInt(puntoT[0].trim());
    int y2 = Integer.parseInt(puntoT[1].trim()); 
\end{lstlisting}
Después el valor de las coordenadas del punto $C$ y del punto $T$ se convierten a números enteros y se calcula la distancia entre ellos utilizando la fórmula de la distancia euclidiana.
\begin{lstlisting}[style=javaStyle]
float distancia = (float)Math.sqrt(Math.pow(x2 - x1, 2) + Math.pow(y2 - y1, 2)); 
\end{lstlisting}

Al final se verifica la ubicación del punto, se comparan la distancia calculada con el radio $r$ de la circunferencia para determinar la ubicación del punto $T$. 
Dependiendo de la comparación, se imprime un mensaje indicando si el punto está dentro de la circunferencia, en el borde o fuera de ella.
\begin{lstlisting}[style=javaStyle]
if (distancia > r) {
     System.out.println("El punto T("+x2+","+y2+") esta fuera de la circunferencia");
     }else if (distancia == r) {
     System.out.println("El punto T("+x2+","+y2+") esta en la circunferencia");
    }else{
     System.out.println("El punto T("+x2+","+y2+") esta dentro de la circunferencia");
    }
\end{lstlisting}


\begin{itemize}
    \item Depuración y pruebas: 
\end{itemize}
\begin{table}[h!]
     \centering
     \caption{Tabla de Corridas del problema 3}\\

     \begin{tabular}{|c|c|c|c|c|c|}
     \hline
    Corrida & Coordenadas $(x_{1}, y_{1})$& Coordenadas $(x_{2}, y_{2})$  &  Radio $r$ & Resultado\\
    \hline
    1  &  $(3,4)$ & $(9,2)$ & 10 & El punto $T$ esta adentro \\
    \hline
    2  &  $(5,2)$ & $(8,1)$ & 12 & El punto $T$ esta adentro \\
    \hline
    3  &  $(34,23)$ & $(-90,35)$ & 67 & El punto $T$ esta afuera \\
    \hline
    4  &  $(-27,3)$ & $(34,-5)$ & 30 & El punto $T$ esta afuera \\
    \hline
    5 &  $(2,8)$ & $(4,9)$ & 17 & El punto $T$ esta adentro \\
    \hline
     \end{tabular}
     \label{tab:my_label}
 \end{table}


\newpage 
Contenido del tercer problema...
\newpage 


\section{Sección Problema 4}
Contenido del cuarto problema...
\newpage 
Contenido del cuarto problema...
\newpage 


\section{Sección Problema 5}
Contenido del quinto problema...
\newpage 
Contenido del quinto problema...
\newpage 


\section{Sección Problema 6}
Contenido del sexto problema...
\newpage 
Contenido del sexto problema...
\newpage 



\end{document}

