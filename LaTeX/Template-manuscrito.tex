\documentclass{IEEEcsmag}

\usepackage[colorlinks,urlcolor=blue,linkcolor=blue,citecolor=blue]{hyperref}
\expandafter\def\expandafter\UrlBreaks\expandafter{\UrlBreaks\do\/\do\*\do\-\do\~\do\'\do\"\do\-}
\usepackage{upmath,color}

\usepackage[spanish]{babel}
%\usepackage[latin1]{inputenc}
\usepackage[utf8]{inputenc}  

\jvol{1}
\jnum{1}
\paper{1}
\jmonth{Noviembre}
\jname{ITICs letters}
\jtitle{Proyectos Integradores}
\pubyear{2023}

\newtheorem{theorem}{Theorem}
\newtheorem{lemma}{Lemma}



\setcounter{secnumdepth}{0}

\begin{document}

\sptitle{Proyecto Integrador de Primer Semestre}

\title{Software de resolución de problemas de Ingeniería }

\author{Cuadros Romero Francisco Javier}
\affil{Instituto Tecnológico Superior del Occidente del Estado de Hidalgo, Mixquiahuala, Hgo., 42700, Mexico}

\author{Neri Pérez Giovany Humberto}
\affil{Instituto Tecnológico Superior del Occidente del Estado de Hidalgo, Mixquiahuala, Hgo., 42700, Mexico}

%\author{Third Author III}
%\affil{Institute, City, (State), Postal Code, Country}

\markboth{ITSOEH/ITICS/PROYECTO INTEGRADOR PRIMER SEMESTRE}{THEME/FEATURE/DEPARTMENT}

\begin{abstract}
Un resumen (abstract) es un párrafo único que resume los aspectos importantes del manuscrito. A menudo indica si el manuscrito es un informe de un trabajo nuevo, una revisión o una descripción general, o una combinación de ambos. No cite referencias en el resumen. Este tipo de documento debe incluir contenido propiedad de los autores; es decir, no debe contener contenido de otras fuentes, ademas la redacción debe  estar dirigida a un tipo de lector técnico general. Este archivo se encuentra disponible en \href{https://github.com/fcuadrosgithub/integrador-primero.git}{https://github.com/fcuadrosgithub/integrador-primero.git}.
\end{abstract}

\maketitle
\chapteri{L}a introducción debe proporcionar información general (incluidas referencias relevantes) y debe indicar el propósito del manuscrito. En esta sección describa de manera clara y precisa el objetivo del proyecto integrador, la metodología que piensa usar y los resultados obtenidos de manera muy general. Dentro de esta sección puede citar trabajos relevantes de otros si lo cree necesario.

Esta sección debe dar un panorama muy general al lector de cual es el problema a resolver, que metodología utilizó para dar solución al problema y cuales fueron los resultados obtenidos. 

La redacción del manuscrito debe ser en tercera persona y queda estrictamente prohibido el uso de palabras coloquiales o Español informal. En lugar de esto utilice un lenguaje formal que el mayor numero de personas pueda entender.

\section{COPYRIGHT Y ACCESO ABIERTO}

Una vez que los autores entreguen este documento para su evaluación también seden los derechos del contenido de este manuscrito a la carrera de Ingeniería en Tecnológicas de la Información y Comunicaciones (ITICs) del Instituto Tecnológico Superior del Occidente del Estado de Hidalgo (ITSOEH). Esto conlleva que la carrera puede usar el contenido de este articulo para efectos de difusión del quehacer de los estudiantes de la carrera o en cualquier otra actividad que la carrera considere pertinente. Cabe mencionar que en ningún momento el orden o los nombres de los autores sera modificado de ninguna manera y siempre se les dará el crédito correspondiente. 
\section{PROBLEMAS}
A continuación se describen los problemas que el equipo deberá resolver.
\begin{enumerate}
\item Dados 2 puntos $A \mbox{ y } B$ con coordenadas $x_{1}, y_{1}$ y $x_{2}, y_{2}$  respectivamente. Regresar la ecuación de la recta y el ángulo interno $\alpha$ que se forma entre el eje horizontal y la recta. 
%Por ejemplo con los puntos $A(2, 1)$ y $B(-3, 2)$ la ecuación debe ser $y = -\frac{1}{5}x + \frac{7}{5}$. 
\item Dada una ecuación cuadratica regresar los valores de las raíces en caso de que estén sobre el conjunto de los números reales, en caso contrario indicar que la solución esta en el conjunto de los números complejos. 
\item Dada una circunferencia con centro en el punto $C$ con coordenadas $(x_{1}, y_{1})$ y radio $r$, evaluar si un punto $T$ con coordenadas $(x_{2}, y_{2})$ esta dentro del area de la circunferencia.
\item Dado un numero decimal entero positivo o negativo regresar su equivalente en binario.
\item Dado un numero binario de $n$ bits regresar su equivalente en decimal.
\item Dada una tabla de verdad de $n$ bits generar la expresión booleana que genere de manera fidedigna las salidas de esta tabla.
\end{enumerate}

\section{Sección Problema 1}
Contenido del primer problema...
\newpage
Contenido del primer problema...
\newpage


\section{Sección Problema 2}
Contenido del segundo problema...
\newpage
Contenido del segundo problema...
\newpage


\section{Sección Problema 3}
Contenido del tercer problema...
\newpage 
Contenido del tercer problema...
\newpage 


\section{Sección Problema 4}
Contenido del cuarto problema...
\newpage 
Contenido del cuarto problema...
\newpage 


\section{Sección Problema 5}
\section{DESCRIPCIÓN DEL PROBLEMA}
El reporte analiza el concepto de ingresar un número de tipo binario con $n$ bits, para posteriormente regresar su equivalente en decimal.

\section{DEFINICIÓN DE SOLUCIÓN}
Se planteó el generar un programa el cual sea capaz de realizar dicha premisa

\section{DISEÑO DE LA SOLUCIÓN}
\begin{enumerate}
    \item Solicitar al usuario que ingrese el número binario de n bits.
    \item Validar que el número binario ingresado sea válido, es decir, que esté compuesto únicamente de 0's y 1's y tenga una longitud de n bits.
    \item Calcular el número decimal equivalente utilizando el método binAdecimal.
    \item Mostrar el resultado al usuario.
\end{enumerate}

\section{DESARROLLO DE LA SOLUCIÓN}
El algoritmo de solución del problema comienza solicitando al usuario teclee el número binario a convertir, para posteriormente almacenarlo en la variable $nbinario$:.
\begin{lstlisting}[style=javaStyle]

Scanner bin = new Scanner(System.in);
    System.out.println("Ingresa el numero binario: ");
    String nbinario = bin.nextLine();
    bin.close();
        
\end{lstlisting}

Por consiguiente, se declara la variable donde se va a alacena1r el dato y el mensaje a imprimir en pantalla de la operación de conversión.

\begin{lstlisting}[style=javaStyle]

int num = binAdecimal(nbinario);
System.out.println("El numero decimal equivalente es: " + num);

\end{lstlisting}

Como el procedimiento de conversión se realizó dentro de una clase privada, se creo una de estas para posteriormente hacer el procedimiento, aquí se declara la clase y se definen las variables con el tipo de dato que estas van a llegar a ser.

\begin{lstlisting}[style=javaStyle]

public static int binAdecimal(String binario){
    int n = binario.length();
    int decimal = 0;
    }

\end{lstlisting}

Luego de haber definido todas nuestras variables, viene el procedimiento de resolución del programa, Para convertir un número binario a decimal, podemos utilizar un bucle "$for$". Primero, inicializamos una variable 'decimal' en 0. Luego, recorremos cada dígito del número binario utilizando la variable $i$ como contador.

\begin{lstlisting}[style=javaStyle]

    for (int i = 0; i < n; i++) {
    }
\end{lstlisting}

En cada iteración del bucle, extraemos el dígito binario correspondiente utilizando la función "$Character.getNumericValue(binario.charAt(i))$". Este paso convierte el carácter binario en un entero.

\begin{lstlisting}[style=javaStyle]
    int bit=Character.getNumericValue
        (binario.charAt(i));
\end{lstlisting}

Luego, multiplicamos este dígito por 2 elevado a la potencia correspondiente. Utilizamos la fórmula "$bit * Math.pow(2, n - 1 - i)$" para realizar esta multiplicación, donde "$bit$" es el dígito binario, $n$ es el tamaño del número binario y "$i$" es el contador del bucle.

\begin{lstlisting}[style=javaStyle]
decimal += bit * Math.pow(2, n - 1 - i);
\end{lstlisting}

Finalmente, sumamos el resultado de cada multiplicación a la variable $decimal$.
Al finalizar el bucle, la variable $decimal$ contendrá el valor decimal equivalente al número binario original.

El código queda de la siguiente manera:\\
\begin{lstlisting}[style=javaStyle]
for (int i = 0; i < n; i++) {
    int bit = Character.getNumericValue
    (binario.charAt(i));
    decimal += bit * Math.pow
    (2, n - 1 - i);
    }
\end{lstlisting}

\section{DEPURACION Y PRUEBAS}
 
 \begin{table}[h]
     \centering
     \caption{Tabla de Corridas}\\
     
     \begin{tabular}{|c|c|c|}
     \hline
        Corrida & Binario & Decimal\\
        \hline
        1  & 010111 & 23\\
        \hline
        2  & 011011 & 27\\
        \hline
        3  & 101010 & 42\\
        \hline
        4  & 010101 & 21\\
        \hline
        5  & 111011 & 59\\
        \hline
     \end{tabular}
     \label{tab:my_label}
 \end{table}

\section{CONCLUSION}
El algoritmo de solución del problema se elaboro con el fin de convertir un número binario ingresado por el usuario a su equivalente decimal. El algoritmo utiliza un bucle "for" para recorrer cada dígito del número binario. En cada iteración, se extrae el dígito binario y se multiplica por 2 elevado a la potencia correspondiente. El resultado de cada multiplicación se suma a una variable decimal. Al finalizar el bucle, la variable decimal contiene el valor decimal equivalente al número binario original. El código muestra el número decimal resultante en la pantalla.


\section{Sección Problema 6}
Contenido del sexto problema...
\newpage 
Contenido del sexto problema...
\newpage 



\end{document}

